% !TeX spellcheck = en_GB
\section*{Mathematical Modelling: Using Markov chains to model bathroom queues}
\vspace{-.30cm}

\title{Mathematical Modelling: Using Markov chains to model bathroom queues}

\begin{center}
	\textbf{Johnny Wong}\footnote{%
		Johnny Wong is a recent graduate of UNSW, Australia ({\tt johnny.c.wong@unswalumni.com})}
\end{center}

\vspace{5mm}

You come home from your morning run all sweaty and ready for a shower. As you approach the bathroom, you see the locked door and roll your eyes. Standing in your sweat soaked singlet, your little brother pokes his head out of his bedroom, chucks a deodorant at you before telling you he had already called dibs on the bathroom after you sister is finished. You throw the deodorant back at him and curse your house for having so few bathrooms.

Everyone has had to wait to use the bathroom at some point in their lives. It's common sense that the fewer bathrooms or more people there are, the longer the expected wait time. But how can we mathematically calculate how long we can expect to wait every day? One approach is with a technique called Markov chains.

\subsection*{Markov chains}
Markov chains model the movement between some \textit{states} throughout \textit{time}, where the \textit{probability} of moving from state to state can be defined.

\subsubsection*{States}
A state is something that can describe the scenario we are interested in. For example, if we are interested in the weather, we could define 3 states: sunny, cloudy, raining.\\
For our problemWe are interested in the 

\subsubsection*{Time}
